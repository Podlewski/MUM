\documentclass[a4paper,11pt]{article}
\usepackage[verbose,a4paper,tmargin=2cm,bmargin=2cm,lmargin=2.5cm,rmargin=2.5cm]{geometry}
\usepackage{polski}
\usepackage{amsmath}
\usepackage{amsfonts}
\usepackage{amssymb}
\usepackage{lastpage}
\usepackage{indentfirst}
\usepackage{verbatim}
\usepackage{graphicx}
\usepackage{fancyhdr}
\usepackage{listings}
\usepackage{float}
\usepackage{hyperref}
\hypersetup{
    colorlinks = true,
    linkcolor = black,
    urlcolor = cyan
}
\usepackage{xcolor}
\usepackage{tikz}
\usepackage{multirow}
\frenchspacing
\pagestyle{fancyplain}
\fancyhf{}

\usepackage{setspace}
\usepackage{enumitem}

\renewcommand{\headrulewidth}{0pt}
\renewcommand{\footrulewidth}{0.4pt}
\newcommand{\degree}{\ensuremath{^{\circ}}} 
\fancyfoot[L]{MUM: P. Galewicz, B. Jurczewski, Z. Nowacki, K.Podlewski, P. Wardęcki}
\fancyfoot[R]{\thepage\ / \pageref{LastPage}}


%%%%%%%%%%%%%%%%%%%%%%%%%%%%%%%%%%%%%%%%%%%% STRONA TYTUŁOWA

\begin{document}


\begin{titlepage}
\begin{center}
\begin{tabular}{rcl}
\begin{tabular}{|r|}
\hline \\
\large{\underline{234053~~~~~~~~~~~~~~~~~~~~~~~} }\\
\small{\textit{Numer indeksu}}\\
\large{\underline{Paweł Galewicz~~~~~~~~~~~~} }\\
\small{\textit{Imię i nazwisko}}\\\\ \hline
\end{tabular} 
&
\begin{tabular}{|r|}
\hline \\
\large{\underline{234067~~~~~~~~~~~~~~~~~~~~~~~} }\\
\small{\textit{Numer indeksu}}\\
\large{\underline{Bartosz Jurczewski~~~~~~~} }\\
\small{\textit{Imię i nazwisko}}\\\\ \hline
\end{tabular} 
&
\begin{tabular}{|r|}
\hline \\
\large{\underline{234102~~~~~~~~~~~~~~~~~~~~~~~} }\\
\small{\textit{Numer indeksu}}\\
\large{\underline{Zbigniew Nowacki~~~~~~~~} }\\
\small{\textit{Imię i nazwisko}}\\\\ \hline
\end{tabular} 
\end{tabular} 

\vspace{10px}

\begin{tabular}{rl}
\begin{tabular}{|r|}
\hline \\
\large{\underline{234106~~~~~~~~~~~~~~~~~~~~~~~} }\\
\small{\textit{Numer indeksu}}\\
\large{\underline{Karol Podlewski~~~~~~~~~~~} }\\
\small{\textit{Imię i nazwisko}}\\\\ \hline
\end{tabular} 
&
\begin{tabular}{|r|}
\hline \\
\large{\underline{234128~~~~~~~~~~~~~~~~~~~~~~~} }\\
\small{\textit{Numer indeksu}}\\
\large{\underline{Piotr Wardęcki~~~~~~~~~~~~} }\\
\small{\textit{Imię i nazwisko}}\\\\ \hline
\end{tabular} 
\end{tabular}
\end{center}

\vspace{25px}

\begin{tabular}{ll}
\LARGE{\textbf{Kierunek}}& \LARGE{Informatyka Stosowana} \\
\LARGE{\textbf{Stopień}}& \LARGE{II} \\
\LARGE{\textbf{Specjalizacja}}& \LARGE{Data Science} \\
\LARGE{\textbf{Semestr}}& \LARGE{1} \\\\
\LARGE{\textbf{Data oddania}}& \LARGE{29 kwietnia 2020} \\\\\\\\\\\\\\
\end{tabular}

\begin{center}
\textbf{\huge{\\~\\Metody uczenia maszynowego}}
\textbf{\Huge{\\~\\Analiza danych przestępczych}}
\textbf{\Large{\\~\\Propozycja projektu grupowego}}
\end{center}

\end{titlepage}

\setcounter{page}{2}
% \setstretch{1.5}
% \tableofcontents
\newpage
\setstretch{1.1}


%%%%%%%%%%%%%%%%%%%%%%%%%%%%%%%%%%%%%%%%%%%% TYTUŁ

\section{Tytuł} \label{sec:tytuł}

Projekt jaki zamierzamy wykonać został zatytułowany \textbf{Analiza danych przestępczych}.

%%%%%%%%%%%%%%%%%%%%%%%%%%%%%%%%%%%%%%%%%%%% DANE

\section{Dane członków grupy} \label{sec:dane}

Skład grupy:

\begin{itemize}[noitemsep,topsep=0pt,leftmargin=1.1cm]
    \item Paweł Galewicz        \quad 234053
    \item Bartosz Jurczewski    \quad 234067
    \item Zbigniew Nowacki      \quad 234102
    \item Karol Podlewski       \quad 234106
    \item Piotr Wardęcki        \quad 234128
\end{itemize}


%%%%%%%%%%%%%%%%%%%%%%%%%%%%%%%%%%%%%%%%%%%% MOTYWACJA

\section{Motywacja} \label{sec:motywacja}

Klasyfikacja danych pozwala połączyć przypadki, które dla ludzkiego oka nie wydają się mieć ze sobą wiele wspólnego. Taka analiza może się okazać niezwykle przydatna nie tylko w przypadku danych medycznych, ale też przestępczych. Wiedząc co łączy nietypowe rabunki czy akty wandalizmu możemy lepiej zaplanować działania prewencyjne, a także lepiej chronić nasze mienie jak i samych siebie.


%%%%%%%%%%%%%%%%%%%%%%%%%%%%%%%%%%%%%%%%%%%% CELE PROJEKTU

\section{Cele projektu}

Celem projektu jest przeprowadzenie badania mającego na celu zwiększenie efektywności działań prewencyjnych służb mundurowych. Program zostanie wykonany przy użyciu języka programowania Python wraz z różnymi dostępnymi frameworkami pozwalającymi na szybką i skuteczną implementację niezbędnych algorytmów. Wykonanie projektu pozwoli nam na lepsze zrozumienie działania algorytmów oraz wzrost umiejętności pracy z dużymi zbiorami danych które są wykorzystywane w rzeczywistych warunkach. Dodatkowymi korzyściami będzie nauka pisania czytelniejszego kodu w składni języka Python, a także lepsze poznanie wielu bibliotek pomocniczych, tylko pośrednio związanych z analizą danych. Wykonanie projektu powinno się zakończyć wyciągnięciem i przedstawieniem uzyskanych wniosków na początku czerwca.

%%%%%%%%%%%%%%%%%%%%%%%%%%%%%%%%%%%%%%%%%%%% JAKIE DANE

\section{Opis zbiorów danych} \label{sec:dataset}

Dane zostaną pozyskane z wykorzystaniem najpopularniejszych repozytoriów danych (m.in. Kaggle, UCI Repository czy Makeover Monday). Zostanie podjęta próba odszukania danych na stronach anglo- i polskojęzycznych, które są bezpośrednio lub pośrednio powiązane z publicznymi jednostkami monitorującymi przestępstwa (na przykład strony rządowe czy strony 
funkcjonariatów policji). Scenariusz optymistyczny zakłada pozyskanie danych zbieranych na przestrzeni wielu lat. Takie dane potencjalnie pomóc mogą w odkryciu interesujących trendów, które pozwolą na głębsze zrozumienie zjawiska przestępczości oraz gdzie może ono zmierzać w przyszłości.
\par
Dodatkowo planowane jest zestawienie danych o występkach ze statystykami miejsc, w których dane zjawiska występują. Poszerzy to pogląd na dane i pozwoli na poszerzenie poglądu na informacje o otoczeniu i społeczności, w której występki mają miejsce.

%%%%%%%%%%%%%%%%%%%%%%%%%%%%%%%%%%%%%%%%%%%% BADANIA

\section{Plan czynności}

\begin{enumerate}
    \item Pozyskanie jak największej liczby danych o przestępczości
    \item Zapoznanie się ze zgromadzonym zestawem i ocena jego użyteczności
    \item Czyszczenie zbioru danych oraz próba wyekstrahowania najistotniejszych cech
    \item Zestawienie danych o przestępczości z danymi statystycznymi miejsc ich występowania
    \item Analiza pod kątem niestandardowych, obiecujących korelacji danych, które warte będą dalszego eksplorowania
    \item Analiza skupień w celu znalezienia wzorców
    \item Analiza częstotliwości oraz wystąpień w czasie, określenie tendencji i próba predykcji przyszłych wartości
    \item Porównanie obszarów z największą częstością występowania zjawisk przestępczości w czasie
\end{enumerate}

\end{document}
