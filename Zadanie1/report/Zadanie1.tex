\documentclass[a4paper,11pt]{article}
\usepackage[verbose,a4paper,tmargin=2cm,bmargin=2cm,lmargin=2.5cm,rmargin=2.5cm]{geometry}
\usepackage[utf8]{inputenc}
\usepackage{polski}
\usepackage{amsmath}
\usepackage{amsfonts}
\usepackage{amssymb}
\usepackage{lastpage}
\usepackage{indentfirst}
\usepackage{verbatim}
\usepackage{graphicx}
\usepackage{fancyhdr}
\usepackage{listings}
\usepackage{hyperref} 
\usepackage{xcolor}
\usepackage{tikz}
\frenchspacing
\pagestyle{fancyplain}
\fancyhf{}

\usepackage{setspace}

\renewcommand{\headrulewidth}{0pt}
\renewcommand{\footrulewidth}{0.4pt}
\newcommand{\degree}{\ensuremath{^{\circ}}} 
\fancyfoot[L]{MUM: P. Galewicz, B. Jurczewski, Z. Nowacki, K.Podlewski, P. Wardęcki}
\fancyfoot[R]{\thepage\ / \pageref{LastPage}}


\begin{document}

\begin{titlepage}
\begin{center}
\begin{tabular}{rcl}
\begin{tabular}{|r|}
\hline \\
\large{\underline{234053~~~~~~~~~~~~~~~~~~~~~~~} }\\
\small{\textit{Numer indeksu}}\\
\large{\underline{Paweł Galewicz~~~~~~~~~~~~} }\\
\small{\textit{Imię i nazwisko}}\\\\ \hline
\end{tabular} 
&
\begin{tabular}{|r|}
\hline \\
\large{\underline{234067~~~~~~~~~~~~~~~~~~~~~~~} }\\
\small{\textit{Numer indeksu}}\\
\large{\underline{Bartosz Jurczewski~~~~~~~} }\\
\small{\textit{Imię i nazwisko}}\\\\ \hline
\end{tabular} 
&
\begin{tabular}{|r|}
\hline \\
\large{\underline{234102~~~~~~~~~~~~~~~~~~~~~~~} }\\
\small{\textit{Numer indeksu}}\\
\large{\underline{Zbigniew Nowacki~~~~~~~~} }\\
\small{\textit{Imię i nazwisko}}\\\\ \hline
\end{tabular} 
\end{tabular} 

\\~\\~\\

\begin{tabular}{rl}
\begin{tabular}{|r|}
\hline \\
\large{\underline{234106~~~~~~~~~~~~~~~~~~~~~~~} }\\
\small{\textit{Numer indeksu}}\\
\large{\underline{Karol Podlewski~~~~~~~~~~~} }\\
\small{\textit{Imię i nazwisko}}\\\\ \hline
\end{tabular} 
&
\begin{tabular}{|r|}
\hline \\
\large{\underline{234128~~~~~~~~~~~~~~~~~~~~~~~} }\\
\small{\textit{Numer indeksu}}\\
\large{\underline{Piotr Wardęcki~~~~~~~~~~~~} }\\
\small{\textit{Imię i nazwisko}}\\\\ \hline
\end{tabular} 
\end{tabular}
\end{center}

\\~\\~\\~\\ 

\begin{tabular}{ll}
\LARGE{\textbf{Kierunek}}& \LARGE{Informatyka Stosowana} \\
\LARGE{\textbf{Stopień}}& \LARGE{II} \\
\LARGE{\textbf{Specjalizacja}}& \LARGE{Data Science} \\
\LARGE{\textbf{Semestr}}& \LARGE{1} \\\\
\LARGE{\textbf{Data oddania}}& \LARGE{18 marca 2020} \\\\\\\\\\\\\\
\end{tabular}

\begin{center}
\textbf{\huge{\\~\\Metody uczenia maszynowego }}
\textbf{\Huge{\\~\\Problem set 1}}
\end{center}

\end{titlepage}

\setcounter{page}{2}
\setstretch{1.5}

\tableofcontents
\newpage

\section{Cel}
Zadanie polega na analizie procesu klasyfikacji danych za pomocą wybranych metod:
\begin{enumerate}
    \item Algorytm drzew decyzyjnych
    \item Naiwny klasyfikator Bayesa
    \item Maszyna wektorów nośnych
    \item Klasyfikator k-najbliższych sąsiadów
    \item Algorytm sztucznych sieci neuronowych
\end{enumerate}

Należy zaimplementować każdą metodę, a następnie zweryfikować jej działanie biorąc pod uwagę:
\begin{itemize}
    \item różne możliwe ustawienia parametrów konfiguracyjnych i ich wpływ na wyniki klasyfikacji
    \item zbiory danych o różnej charakterystyce (przynajmniej 3 różne zbiory)
\end{itemize}
Każdą metodę należy przetestować na tych samych zbiorach, a następnie porównać wyniki i wyciągnąć wnioski dotyczące skuteczności poszczególnych metod. Jako kryterium porównawcze wystarczy omówić dokładność klasyfikacji (accuracy), pozostałe kryteria są opcjonalne.

\section{Wprowadzenie}
\subsection{Algorytm drzew decyzyjnych}
Opis

\subsection{Naiwny klasyfikator Bayesa}
Opis

\subsection{Maszyna wektorów nośnych}
Opis

\subsection{Klasyfikator k-najbliższych sąsiadów}
Opis

\subsection{Algorytm sztucznych sieci neuronowych}
Opis

\section{Opis implementacji}
Algorytmy zostały zaimplementowane za pomocą języka Python w wersji 3.8.2.
Wykorzystano w nim biblioteki NumPy, Sklearn i Pandas. Bazowaliśmy na trzech zestawach danych: 
\begin{itemize}
    \item{Fall Detection Data from China - \url{https://www.kaggle.com/pitasr/falldata}}
    \item{Rain in Australia - \url{https://www.kaggle.com/jsphyg/weather-dataset-rattle-package}}
    \item{Suicide Rates Overview 1985 to 2016 - \url{https://www.kaggle.com/russellyates88/suicide-rates-overview-1985-to-2016}}
\end{itemize}

\section{Badania}
\color{red}
Cytuję: "Należy zaimplementować każdą metodę, a następnie zweryfikować jej działanie biorąc pod uwagę:\\
A. różne możliwe ustawienia parametrów konfiguracyjnych i ich wpływ na wyniki klasyfikacji"
B. zbiory danych o różnej charakterystyce (przynajmniej 3 różne zbiory)
\color{black}

\subsection{Algorytm drzew decyzyjnych}
\subsubsection{Różne ustawienia parametrów konfiguracyjnych}
\subsubsection{Różne zbiory danych}

\subsection{Naiwny klasyfikator Bayesa}
\subsubsection{Różne ustawienia parametrów konfiguracyjnych}
\subsubsection{Różne zbiory danych}

\subsection{Maszyna wektorów nośnych}
\subsubsection{Różne ustawienia parametrów konfiguracyjnych}
\subsubsection{Różne zbiory danych}

\subsection{Klasyfikator k-najbliższych sąsiadów}
\subsubsection{Różne ustawienia parametrów konfiguracyjnych}
\subsubsection{Różne zbiory danych}

\subsection{Algorytm sztucznych sieci neuronowych}
\subsubsection{Różne ustawienia parametrów konfiguracyjnych}
\subsubsection{Różne zbiory danych}


\end{document}